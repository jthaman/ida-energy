% Created 2016-12-08 Thu 09:14ad
% Intended LaTeX compiler: pdflatex
\documentclass[presentation]{beamer}
\usepackage[english]{babel}
\usepackage{subcaption}
\usepackage{graphicx}
\usepackage{grffile}
\usepackage{longtable}
\usepackage{wrapfig}
\usepackage{rotating}
\usepackage[normalem]{ulem}
\usepackage{amsmath}
\usepackage{textcomp}
\usepackage{amssymb}
\usepackage{multicol}
\usepackage{capt-of}
\usepackage{hyperref}
\newenvironment{Figure}
{\par\medskip\noindent\minipage{\linewidth}}
{\endminipage\par\medskip}
\usetheme{metropolis}
\author{John Haman}
\institute{Bowling Green State University, Institute for Defense Analyses}
\date{09 Jan 2018}
\title{The Energy of Data}
\begin{document}
\maketitle

\begin{frame}
  \frametitle{Slides and Code}
  \centering
  \texttt{github.com/jthaman}
\end{frame}

\begin{frame}
  Assumptions about data dominate statistical practice
  \begin{block}{Question:}
    Is the data inconsistent with my hypothesis??
  \end{block}
\end{frame}


\begin{frame}
  \frametitle{What is Data Energy?}
  \begin{centering}
    \begin{block}{In a nutshell \ldots}
      Energy statistics ($\mathcal{E}$-statistics) are functions of distances between statistical
      observations. The value of the $\mathcal{E}$-statistic represents the (potential)
      energy of the data.
    \end{block}
    % Energy statistics (E-statistics) are functions of distances
    % between statistical observations. The value of the energy
    % statistics for a given data set is the (potential) energy of the
    % data. This concept is based on the notion of Newton’s
    % gravitational potential energy, which is a function of the
    % distance between two bodies. The idea of energy statistics is to
    % consider statistical observations as heavenly bodies governed by a
    % statistical potential energy, which is zero if and only if an
    % underlying statistical hypothesis is true.
  \end{centering}
  \pause
  \begin{center}
    Why \textit{distances} ??? 
  \end{center}
  % Classical statistics operated with real-valued data such as height,
  % weight, or blood pressure, and it was supposed that typical data are
  % (approximately) normally distributed so that one can apply the theory
  % of normal (Gaussian) distributions for inference. Even if the
  % observations were not Gaussian, in case of big data when the number of
  % observations, n, was large, one could often refer to central limit
  % theorems to claim that for large n, the normal approximation is valid
  % and classical methods are applicable.  What happens if the data are
  % not real numbers but vectors, functions, graphs, and so on? In this
  % case even addition or multiplication of data might be a problem if,
  % for example, the observed vectors have different dimensions. We can
  % overcome this difficulty if the observations are elements of a metric
  % space. In this case, instead of working with the observations
  % themselves, we can work with their (nonnegative) real-valued
  % distances. This brings us back to the real world where we can work
  % with real numbers. We call this type of inference energy inference.
\end{frame}

\begin{frame}
  \begin{block}{Energy Distance}
    \begin{center}
      $\mathcal{E}(X,Y) = 2\mathbb{E}|X - Y| - \mathbb{E}|X - X'| -
      \mathbb{E}|Y - Y'|$
    \end{center}
  \end{block}
  $|\cdot|$ can be (almost) any measure of distance that you like.
\end{frame}

\begin{frame}{The Energy Inequality}
  \begin{enumerate}
  \item Energy Distance $\mathcal{E}(X,Y) \geq 0$ 
  \item Energy Distance $\mathcal{E}(X,Y) = 0$ if and only if the
    hypothesis is true.
  \end{enumerate}
\end{frame}

\begin{frame}{An old conjecture}
  \begin{block}{Walter Deuber (1998) -- }
    \textit{``For equal numbers of black and white points in euclidean
      space, the sum of the pairwise distances between points of equal
      color is less than or equal to the sum of the pairwise distances
      between points of different color''} \vskip 0.5 in
    \textit{``Equality holds only in the case when black and white
      points coincide''}
  \end{block}
\end{frame}

\begin{frame}
  \frametitle{Visualizing Energy Inequality}
  \begin{center}
    \includegraphics[width = \linewidth]{dots0}
  \end{center}
\end{frame}

\begin{frame}
  \frametitle{Visualizing Energy Inequality}
  \begin{center}
    \includegraphics[width = \linewidth]{dots1}
  \end{center}
\end{frame}


\begin{frame}
  \frametitle{Visualizing Energy Inequality}
  \begin{center}
    \includegraphics[width = \linewidth]{dots2}
  \end{center}
\end{frame}


\begin{frame}
  \frametitle{Visualizing Energy Inequality}
  \begin{center}
    \includegraphics[width = \linewidth]{dots3}
  \end{center}
\end{frame}


\section{One Sample Energy Statistics}

\begin{frame}
  \frametitle{Is the data bell shaped?}
  \includegraphics[width = \linewidth]{hist1}
\end{frame}

\begin{frame}
  \frametitle{Still true?}
  \includegraphics[width = \linewidth]{hist2} 
\end{frame}

\begin{frame}
  \frametitle{ECDF Tests}
\includegraphics[width = \linewidth]{ecdf}
\end{frame}

\begin{frame}{ECDF Tests}
\includegraphics[width = \linewidth]{ks_one_sample}
\end{frame}

\begin{frame}
  \frametitle{ECDF Tests}
\includegraphics[width = \linewidth]{ad_one_sample}  
\end{frame}

% \begin{frame}
%   \frametitle{Problems?!}
%   \begin{itemize}
%   \item How to extend this to multivariate data?
%   \end{itemize}
% \end{frame}

\begin{frame}
  \frametitle{Energy GOF}
  \begin{block}{Energy distance:}
    \begin{center}
      $\mathcal{E}(X,Y) = 2\mathbb{E}|X - Y| - \mathbb{E}|X - X'| -
      \mathbb{E}|Y - Y'|$
    \end{center}
  \end{block}
  replace $Y$ (distribution) with $Y$ (sample version)
  \begin{block}{Estimated Energy distance:}
    \begin{center}
      $$\hat{\mathcal{E}}_n(X, y_1, \ldots, y_n) = \frac{2}{n} \sum_{i=1}^n \mathbb{E}|X - y_i|  - \mathbb{E}|X - X'|  - \frac{1}{n^2} \sum_{i=1}^n \sum_{j=1}^n |y_i - y_j|$$
    \end{center}
  \end{block}  
  % \pause
  % \begin{block}{Notes}
  %   \begin{itemize}
  %   \item If $F = F_0$, then $n \mathcal{E}_n(\bold{X},F_0)$ converges to a distribution.
  %   \item If not, then $n \mathcal{E}_n$ goes to infinity. 
  %   \item $\sqrt{\mathcal{E}_n}$ is a metric (on samples of size $n$)
  %   \end{itemize}
  % \end{block}
\end{frame}

\section{Application to Weibull Distribution}

\begin{frame}
  \frametitle{Motivation}
  \begin{Figure}
    \centering
    \includegraphics[width=\linewidth]{stryker}
    \captionof{figure}{Stryker}
  \end{Figure}
\end{frame}

\begin{frame}
  \includegraphics[width = \textwidth]{rebecca}
\end{frame}

\begin{frame}
  \frametitle{Motivation}
  \begin{Figure}
    \centering
    \includegraphics[width=\linewidth]{pp_plot}
    \captionof{figure}{Source: IDA NS-D-5137}
  \end{Figure}
\end{frame}

\begin{frame}
  \frametitle{Weibull Energy}
  \begin{displaymath}
    \mathbb{E}|X-y_i| = y_i - 2y_i \exp\left( - \left(\frac{y_i}{b}\right)^a \right) + b
    \Gamma \left(1 + \frac{1}{a}\right) -2b \gamma_{*} \left(1 + \frac{1}{a}, \left( \frac{y_i}{b} \right)^a\right)
  \end{displaymath}
  \vskip 0.5in
  \begin{displaymath}
    \mathbb{E}|X - X'| = 2b\Gamma \left( 1 + \frac{1}{a} \right)
    \left( 1 - \frac{1}{\sqrt[a]{2}} \right)
  \end{displaymath}
\end{frame}

\begin{frame}
  \frametitle{Energy Compared to other Tests}
  \begin{figure}
    \centering
    \includegraphics[width = \linewidth]{weibull_power_shape}
  \end{figure}
\end{frame}

\begin{frame}
  \frametitle{Energy Compared to other Tests}
  \begin{figure}
    \centering
    \includegraphics[width = \linewidth]{weibull_power}
  \end{figure}
\end{frame}

\begin{frame}
  \frametitle{Energy Test P-values under Null Hypothesis}
  \begin{figure}
    \centering
    \includegraphics[width = \linewidth]{pvals}
  \end{figure}
\end{frame}


\begin{frame}
 \includegraphics[width = \textwidth]{weibulls1} 
\end{frame}


\begin{frame}
 \includegraphics[width = \textwidth]{weibulls2} 
\end{frame}

\begin{frame}
  \frametitle{Wait this isn't realistic}
  \begin{block}{My hypothesis}
    My data is Weibull with known parameters
  \end{block}

  \begin{block}{My desired hypothesis}
    My data is Weibull with \textbf{unknown} parameters
  \end{block}
\end{frame}

\section{Two (or more) Sample Problems}

\begin{frame}
\includegraphics[width = \linewidth]{scatter}
\end{frame}

\begin{frame}
  \frametitle{KS Distance}
  \includegraphics[width = \linewidth]{ks_two_sample}
\end{frame}

\begin{frame}
  \frametitle{Two Sample Energy}
  \begin{block}{Energy Distance:}
    \begin{center}
      $\mathcal{E}(X,Y) = 2\mathbb{E}|X - Y| - \mathbb{E}|X - X'| -
      \mathbb{E}|Y - Y'|$
    \end{center}
  \end{block}
  replace $X$ and $Y$ (distribution) with $X$ and $Y$ (sample version)
  \begin{block}{Estimated Energy Distance:}
    \begin{gather*}
      \hat{\mathcal{E}}_{n_1, n_2} = \\ \frac{2}{n_1 n_2} \sum \sum |x_i - y_k| -
      \frac{1}{n_1} \sum \sum |x_i - x_j| - \frac{1}{n_2} \sum \sum |y_\ell - y_k| 
    \end{gather*}

    \begin{displaymath}
      = \textsf{2 *(avg ''between'' distance) - (avg ''within'' distances)}
    \end{displaymath}
  \end{block}
\end{frame}

\section{Application to Simulation Validation}

\begin{frame}
  \includegraphics[width = \textwidth]{kelly}
\end{frame}

\begin{frame}
  \includegraphics[width = \linewidth]{kelly2}
\end{frame}

\begin{frame}
  \frametitle{Energy vs. KS}
 \includegraphics[width = \textwidth]{ks_energy_means} 
\end{frame}


\begin{frame}
  \frametitle{Energy vs. KS}
 \includegraphics[width = \textwidth]{ks_energy_sds} 
\end{frame}


\begin{frame}{Many Other Applications!}
  \fontsize{7pt}{7.2}\selectfont
  \begin{tabular}[c]{lll}
    \hline
    & \textbf{Classical Approach} & \textbf{Energy Approach} \\
    \hline
    \\
    Dependence & Pearson's Correlation & Distance Correlation \\
    \\
    \hline
    \\
    Goodness-of-fit & EDF tests & One-Sample Energy test \\
    \\
    \hline
    \\
    Multivariate Normality & Skewness and Kurtosis & Multivariate Energy test for Normality \\
    \\
    \hline
    \\
    Multisample Problems & ANOVA & Distance Components (DISCO) \\
    \\
    \hline
    \\
    Cluster Analysis & Ward's Method or $k$-Means & Hierarchical Energy Clustering or $k$-Groups\\
    \\
    \hline
  \end{tabular}
  \begin{block}{}
    Many Energy tests can be created to test very specific hypotheses.
  \end{block}
\end{frame}

\section{Why is Energy ``Special''? }

\begin{frame}
  \frametitle{Why is Energy so special?}
  \begin{center}
    \centering
    \includegraphics[width = \linewidth]{triangle}
    % Following Székely (1989), we can represent the three pairs of
    % dualities between energy, matter, and mind with the help of a triangle
    % (Figure 1) whose vertices are energy, matter, and mind, and the
    % connecting sides represent the equivalence/duality/dichotomy between
    % these notions.  Manifestations of matter include mass (m) and disorder
    % (measured by entropy S). Manifestations of the immaterial mind are
    % memory, information (I ), observation, data, and inputs passed on by
    % the sensory organs.  The duality between E and m is Einstein’s famous
    % E = mc 2 (Einstein 1905). The duality between matter and mind is
    % Szilárd’s idea, which first appeared in his 1922 dissertation, that in
    % a closed material system, the decrease of uncertainty/entropy (S)
    % corresponds to the increase of information (I ) in our mind (Szilárd
    % 1929, Schrödinger 1944, Brillouin 2004). To use Szilárd’s words, it is
    % possible to reduce the entropy of a thermodynamic system by the
    % intervention of intel- ligent beings, for example a “Maxwell’s demon.”
    % Thus Szilárd eradicated the ancient dichotomy of matter and mind just
    % as Einstein eradicated the dichotomy of energy and matter. This review
    % is about the third side of the triangle, the connection between energy
    % and mind in terms of data distance D defined below. Our mind regulates
    % the flow of information (I ) and data distance (D), the source of
    % statistical energy, to help achieve mental harmony.  For data X = X 1
    % , . . . , X n and Y = Y 1 , . . . , Y n in Euclidean space
  \end{center}
\end{frame}

\begin{frame}{Summary}
  \begin{block}{Take aways}
    \begin{enumerate}
    \item $\mathcal{E}$ statistics are functions of distances between data.
    \item $\mathcal{E}$ statistics are possible alternatives to
      current IDA practices
    \item Energy methods are powerful, consistent, and practical.
    \end{enumerate}
  \end{block}
\end{frame}


\begin{frame}{Further Reading}
  \begin{center}
    \textit{Thank you very much for your time today.}
  \end{center}

  \begin{block}{Some references}
    \begin{enumerate}
    \item \textit{Energy Statistics}, Chapman \&  Hall / CRC:
      available 2018 (Maybe).
    \item \textit{The Energy of Data}, ARISA: available 2017.
    \item \texttt{energy} package for \texttt{R}: available on CRAN.
    \item Gabor J. Szekely and Maria L. Rizzo (2009). \textit{Brownian Distance Covariance},
      Annals of Applied Statistics, Vol. 3, No. 4, 1236-1265. 
    \item Szekely, G. J. and Rizzo, M. L. (2005) \textit{A New Test for Multivariate Normality},
      Journal of Multivariate Analysis, 93/1, 58-80.
    \end{enumerate}
  \end{block}
\end{frame}

\end{document}
%%% Local Variables:
%%% mode: latex
%%% TeX-master: t
%%% End:
