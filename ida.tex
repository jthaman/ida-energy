% Created 2016-12-08 Thu 09:14ad
% Intended LaTeX compiler: pdflatex
\documentclass[presentation]{beamer}
\usepackage[english]{babel}
\usepackage{xcolor}
\usepackage{subcaption}
\usepackage{graphicx}
\usepackage{grffile}
\usepackage{longtable}
\usepackage{wrapfig}
\usepackage{rotating}
\usepackage[normalem]{ulem}
\usepackage{amsmath}
\usepackage{textcomp}
\usepackage{amssymb}
\usepackage{multicol}
\usepackage{capt-of}
\usepackage{hyperref}
\newenvironment{Figure}
{\par\medskip\noindent\minipage{\linewidth}}
{\endminipage\par\medskip}
\usetheme{metropolis}
\metroset{titleformat=smallcaps}
\author{John Haman}
\institute{Bowling Green State University, Institute for Defense Analyses}
\date{23 Jan 2018}
\title{The Energy of Data}
\begin{document}
\maketitle

\begin{frame}
  \frametitle{Slides and Code}
  \centering
  \includegraphics[scale = 0.3]{cat}\\
  \centering
  \texttt{github.com/jthaman/ida-energy}
\end{frame}

\begin{frame}
  Assumptions about data dominate statistical practice
  \begin{block}{Question:}
    Is the data inconsistent with my hypothesis??
    % Why does IDA care about this question?
  \end{block}
\end{frame}

\begin{frame}
  \frametitle{What is Data Energy?}
  \begin{centering}
    \begin{block}{In a nutshell}
      Energy statistics ($\mathcal{E}$-statistics) are functions of
      distances between statistical observations. The value of the
      $\mathcal{E}$-statistic represents the (potential) energy of the
      data.
    \end{block}
  \end{centering}
  \pause
  \begin{center}
    Why \textit{distances} ??? 
  \end{center}
\end{frame}


\begin{frame}
  \begin{block}{Energy Distance}
    \begin{center}
      $\mathcal{E}(X,Y) = 2\mathbb{E}|X - Y| - \mathbb{E}|X - X'| -
      \mathbb{E}|Y - Y'| \geq 0$
    \end{center}
  \end{block}
  \centering
  $|\cdot|$ is probably Euclidean (straight line) distance.
\end{frame}


\begin{frame}{An old conjecture}
  \begin{block}{Walter Deuber (1998) -- }
    \textit{``For equal numbers of black and white points in euclidean
      space, the sum of the pairwise distances between points of equal
      color is less than or equal to the sum of the pairwise distances
      between points of different color''} \vskip 0.5 in
    \textit{``Equality holds only in the case when black and white
      points coincide''}
  \end{block}
\end{frame}

\begin{frame}
  \frametitle{Visualizing Energy Inequality}
  \begin{center}
    \includegraphics[width = \linewidth]{dots0}
  \end{center}
\end{frame}

\begin{frame}
  \frametitle{Visualizing Energy Inequality}
  \begin{center}
    \includegraphics[width = \linewidth]{dots1}
  \end{center}
\end{frame}


\begin{frame}
  \frametitle{Visualizing Energy Inequality}
  \begin{center}
    \includegraphics[width = \linewidth]{dots2}
  \end{center}
\end{frame}


\begin{frame}
  \frametitle{Visualizing Energy Inequality}
  \begin{center}
    \includegraphics[width = \linewidth]{dots3}
  \end{center}
\end{frame}


\section{One Sample Goodness-of-Fit}

\begin{frame}
  \frametitle{Is the data bell shaped?}
  \includegraphics[width = \linewidth]{hist1}
\end{frame}

\begin{frame}
  \frametitle{Still true?}
  \includegraphics[width = \linewidth]{hist2} 
\end{frame}

\begin{frame}
  \frametitle{ECDF Tests}
  \includegraphics[width = \linewidth]{ecdf}
\end{frame}

\begin{frame}{ECDF Tests}
  \includegraphics[width = \linewidth]{ks_one_sample}
\end{frame}

\begin{frame}
  \frametitle{ECDF Tests}
  \includegraphics[width = \linewidth]{ad_one_sample}  
\end{frame}

% \begin{frame}
%   \frametitle{Problems?!}
%   \begin{itemize}
%   \item How to extend this to multivariate data?
%   \end{itemize}
% \end{frame}

\begin{frame}
  \frametitle{Energy GOF}
  \begin{block}{Energy distance:}
    \begin{center}
      $\mathcal{E}(X,Y) = 2\mathbb{E}|X - Y| - \mathbb{E}|X - X'| -
      \mathbb{E}|Y - Y'|$
    \end{center}
  \end{block}
  replace $Y$ (distribution) with $Y$ (sample version)
  \begin{block}{Estimated Energy distance:}
    \begin{center}
      $$\hat{\mathcal{E}}_n(X, y_1, \ldots, y_n) = \frac{2}{n} \sum_{i=1}^n \mathbb{E}|X - y_i|  - \mathbb{E}|X - X'|  - \frac{1}{n^2} \sum_{i=1}^n \sum_{j=1}^n |y_i - y_j|$$
    \end{center}
  \end{block}  
  % \pause
  % \begin{block}{Notes}
  %   \begin{itemize}
  %   \item If $F = F_0$, then $n \mathcal{E}_n(\bold{X},F_0)$ converges to a distribution.
  %   \item If not, then $n \mathcal{E}_n$ goes to infinity. 
  %   \item $\sqrt{\mathcal{E}_n}$ is a metric (on samples of size $n$)
  %   \end{itemize}
  % \end{block}
\end{frame}

\section{Application to Weibull Distribution}

\begin{frame}
  \frametitle{Motivation}
  \begin{Figure}
    \centering
    \includegraphics[width=\linewidth]{stryker}
    \captionof{figure}{Stryker}
  \end{Figure}
\end{frame}

\begin{frame}
  \includegraphics[width = \textwidth]{rebecca}
\end{frame}

\begin{frame}
  \frametitle{Motivation}
  \begin{Figure}
    \centering
    \includegraphics[width=\linewidth]{pp_plot}
    \captionof{figure}{Source: IDA NS-D-5137}
  \end{Figure}
\end{frame}

\begin{frame}
  \frametitle{Weibull Energy}
  \begin{block}{Estimated Energy distance:}
    \begin{center}
      $$\hat{\mathcal{E}}_n(X, y_1, \ldots, y_n) = \frac{2}{n} \sum_{i=1}^n \textcolor{blue}{\mathbb{E}|X - y_i|}  - \textcolor{red}{\mathbb{E}|X - X'|}  - \frac{1}{n^2} \sum_{i=1}^n \sum_{j=1}^n |y_i - y_j|$$
    \end{center}
  \end{block}  
  \begin{gather*}
    \textcolor{blue}{\mathbb{E}|X-y_i|} = 2y_i F_{weibull}(y_i;a,b)
    -y_i \,\, + \\ b\Gamma\left(1 + \frac{1}{a}\right)  \left( 1 -2
      F_{gamma}\left( \left( \frac{y_i}{b} \right)^a; 1 + \frac{1}{a};
        1\right) \right) 
  \end{gather*}
  \vskip 0.3in
  \begin{displaymath}
    \textcolor{red}{\mathbb{E}|X - X'|} = 2b\Gamma \left( 1 + \frac{1}{a} \right)
    \left( 1 - \frac{1}{\sqrt[a]{2}} \right)
  \end{displaymath}
\end{frame}

\begin{frame}
  \frametitle{Energy Compared to other Tests}
  \begin{figure}
    \centering
    \includegraphics[width = \linewidth]{weibull_power_shape}
  \end{figure}
\end{frame}

\begin{frame}
  \frametitle{Energy Compared to other Tests}
  \begin{figure}
    \centering
    \includegraphics[width = \linewidth]{weibull_power}
  \end{figure}
\end{frame}

\begin{frame}
  \frametitle{Energy Test P-values under Null Hypothesis}
  \begin{figure}
    \centering
    \includegraphics[width = \linewidth]{pvals}
  \end{figure}
\end{frame}


\begin{frame}
  \includegraphics[width = \textwidth]{weibulls1} 
\end{frame}


\begin{frame}
  \includegraphics[width = \textwidth]{weibulls2} 
\end{frame}

\begin{frame}
  \frametitle{Wait this isn't realistic}
  \begin{block}{My hypothesis}
    My data is Weibull with known parameters
  \end{block}
  \vskip 0.5in
  \begin{block}{My desired hypothesis}
    My data is Weibull with \textbf{unknown} parameters
  \end{block}
\end{frame}

\section{Two (or more) Sample Problems}

\begin{frame}
  \includegraphics[width = \linewidth]{scatter}
\end{frame}

\begin{frame}
  \frametitle{KS Distance}
  \includegraphics[width = \linewidth]{ks_two_sample}
\end{frame}

\begin{frame}
  \frametitle{Two Sample Energy}
  \begin{block}{Energy Distance:}
    \begin{center}
      $\mathcal{E}(X,Y) = 2\mathbb{E}|X - Y| - \mathbb{E}|X - X'| -
      \mathbb{E}|Y - Y'|$
    \end{center}
  \end{block}
  replace $X$ and $Y$ (distribution) with $X$ and $Y$ (sample version)
  \begin{block}{Estimated Energy Distance:}
    \begin{gather*}
      \hat{\mathcal{E}}_{n_1, n_2} = \\ \frac{2}{n_1 n_2} \sum \sum |x_i - y_k| -
      \frac{1}{n_1^2} \sum \sum |x_i - x_j| - \frac{1}{n_2^2} \sum \sum |y_\ell - y_k| 
    \end{gather*}
    
    \begin{displaymath}
      = \textsf{2 (avg ``between'' distance) - (avg ``within'' distances)}
    \end{displaymath}
  \end{block}
\end{frame}

\begin{frame}
  \frametitle{Two Sample Energy}
  \begin{center}
    \includegraphics[width = \linewidth]{dots3}
  \end{center}
\end{frame}

\section{Application to Simulation Validation}

\begin{frame}
  \includegraphics[width = \textwidth]{kelly}
\end{frame}

\begin{frame}
  \includegraphics[width = \linewidth]{kelly2}
\end{frame}

\begin{frame}
  \frametitle{Energy vs. KS}
  \includegraphics[width = \textwidth]{ks_energy_means} 
\end{frame}


\begin{frame}
  \frametitle{Energy vs. KS}
  \includegraphics[width = \textwidth]{ks_energy_sds} 
\end{frame}


\begin{frame}{Many Other Applications!}
  \fontsize{7pt}{7.2}\selectfont
  \begin{tabular}[c]{lll}
    \hline
    & \textbf{Classical Approach} & \textbf{Energy Approach} \\
    \hline
    \\
    Dependence & Pearson's Correlation & Distance Correlation \\
    \\
    \hline
    \\
    Goodness-of-fit & EDF tests & One-Sample Energy test \\
    \\
    \hline
    \\
    Multivariate Normality & Skewness and Kurtosis & Multivariate Energy test for Normality \\
    \\
    \hline
    \\
    Multisample Problems & ANOVA & Distance Components (DISCO) \\
    \\
    \hline
    \\
    Cluster Analysis & Ward's Method or $k$-Means & Hierarchical Energy Clustering or $k$-Groups\\
    \\
    \hline
  \end{tabular}
  \begin{block}{}
    Many Energy tests can be created to test very specific hypotheses.
  \end{block}
\end{frame}

\section{Why is Energy ``Special''? }

\begin{frame}{Summary}
  \begin{block}{Take aways}
    \begin{enumerate}
    \item $\mathcal{E}$ statistics are functions of distances between data.
    \item $\mathcal{E}$ statistics are possible alternatives to common
      frequentist methods.
    \item Energy methods are powerful, consistent, and practical.
    \end{enumerate}
  \end{block}
\end{frame}

% \begin{frame}{Further Reading}
%   \begin{center}
%     \textit{Thank you very much for your time today.}
%   \end{center}

%   \begin{block}{Some references}
%     \begin{enumerate}
%     \item Gabor Szekely and Maria Rizzo, \textit{The Energy of Data}, ARISA (2017).
%     \item \texttt{energy} package for \texttt{R}: available on CRAN.
%     \end{enumerate}
%   \end{block}
% \end{frame}


\end{document}
%%% Local Variables:
%%% mode: latex
%%% TeX-master: t
%%% End:
